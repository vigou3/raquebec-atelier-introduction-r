%%% Texte du contrat de licence au début des diapos

\begin{frame}[t,plain,fragile=singleslide]
  \tiny
  \vspace*{10mm}

  \begin{adjustwidth}{15mm}{15mm}
    {\textcopyright} {\year} Vincent Goulet, David Beauchemin, Samuel
    Cabral Cruz \\[4mm]

    \includegraphics[height=4mm,keepaspectratio=true]{by-sa} \\%

    Cette création est mise à disposition selon le contrat
    \href{http://creativecommons.org/licenses/by-sa/4.0/deed.fr}{%
      Attribution-Partage dans les mêmes conditions 4.0 International}
    de Creative Commons. En vertu de ce contrat, vous êtes libre de:
    \begin{itemize}
    \item \textbf{partager} --- reproduire, distribuer et communiquer
      l'{\oe}uvre;
    \item \textbf{remixer} --- adapter l'{\oe}uvre;
    \item utiliser cette {\oe}uvre à des fins commerciales.
    \end{itemize}
    Selon les conditions suivantes: \vspace*{2mm}

    \begin{tabularx}{\linewidth}{@{}lX@{}}
      \raisebox{-5.5mm}[0mm][10mm]{%
      \includegraphics[height=7mm,keepaspectratio=true]{by}} &
      \textbf{Attribution} --- Vous devez créditer l'{\oe}uvre, intégrer
      un lien vers le contrat et indiquer si des modifications ont été
      effectuées à l'{\oe}uvre. Vous devez indiquer ces informations par
      tous les moyens possibles, mais vous ne pouvez suggérer que
      l'Offrant vous soutient ou soutient la façon dont vous avez
      utilisé son {\oe}uvre. \\
      \raisebox{-5.5mm}{\includegraphics[height=7mm,keepaspectratio=true]{sa}}
      & \textbf{Partage dans les mêmes conditions} --- Dans le cas où
      vous modifiez, transformez ou créez à partir du matériel composant
      l'{\oe}uvre originale, vous devez diffuser l'{\oe}uvre modifiée
      dans les même conditions, c'est à dire avec le même contrat avec
      lequel l'{\oe}uvre originale a été diffusée.
    \end{tabularx}
    \vspace{5mm}

    \textbf{Code source} \\
    \viewsource{https://github.com/vigou3/raquebec-intro/}
  \end{adjustwidth}
\end{frame}

%%% Local Variables:
%%% mode: latex
%%% TeX-engine: xetex
%%% TeX-master: "raquebec-atelier-introduction-r"
%%% End:
